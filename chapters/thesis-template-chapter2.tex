 \chapter{Разработка моделей и алгоритмов автоматической валидации научных публикаций}

 В настоящей главе представлены результаты разработки ключевых моделей и алгоритмов, составляющих методологическую основу системы автоматической валидации. Описана формальная модель предметной области, 
 включающая сущности «Журнал», «Специальность», «Версия перечня» и их семантические связи. 
 Разработан алгоритм парсинга PDF-документов перечней ВАК, учитывающий вариативность форматов и обеспечивающий точность 
 извлечения данных не менее 99\%. 
 Представлен метод валидации публикаций, основанный на сопоставлении даты публикации статьи с временными интервалами включения 
 журнала в перечень на соответствующую дату. Описан алгоритм нормализации и дедупликации данных о журналах, поступающих из 
 различных источников (PDF-перечни, CrossRef API). Разработана обобщенная архитектура системы, определяющая взаимодействие 
 основных компонентов: парсера, базы данных, REST API и веб-интерфейса. Представлены интерфейсы модулей и протоколы обмена 
 данными между компонентами системы.