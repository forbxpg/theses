\chapter{Разработка моделей и алгоритмов автоматической валидации научных публикаций}

В настоящей главе представлены результаты разработки ключевых моделей и алгоритмов, составляющих методологическую основу системы автоматической валидации. Описана формальная модель предметной области, 
включающая сущности «Журнал», «Специальность», «Версия перечня» и их семантические связи. 
Разработан алгоритм парсинга PDF-документов перечней ВАК, учитывающий вариативность форматов и обеспечивающий точность 
извлечения данных не менее 99\%. 
Представлен метод валидации публикаций, основанный на сопоставлении даты публикации статьи с временными интервалами включения 
журнала в перечень на соответствующую дату. Описан алгоритм нормализации и дедупликации данных о журналах, поступающих из 
различных источников (PDF-перечни, CrossRef API). Разработана обобщенная архитектура системы, определяющая взаимодействие 
основных компонентов: парсера, базы данных, REST API и веб-интерфейса. Представлены интерфейсы модулей и протоколы обмена 
данными между компонентами системы.

\section{Алгоритмы парсинга данных из PDF-документов}

\begin{annotation}
	Раздел посвящён разработке алгоритмов автоматического извлечения данных из PDF-документов перечней ВАК. 
	Описаны подходы к парсингу табличной информации с учётом вариативности форматов PDF-документов. Представлены алгоритмы 
	распознавания структуры таблиц, извлечения текстовых данных о журналах (ISSN, названия, специальности), обработки 
	многостраничных документов и нормализации извлечённых данных. Описаны методы обработки различных форматов представления 
	ISSN, распознавания специальностей и их кодов, извлечения временных интервалов включения журналов в перечень. 
	Представлены алгоритмы валидации и очистки данных для обеспечения точности парсинга не менее 99\%.
\end{annotation}

\section{Концептуальная модель базы данных}

\begin{annotation}
	Раздел описывает концептуальную модель базы данных для системы автоматической валидации научных публикаций. 
	Представлена трёхуровневая архитектура данных, включающая слой версий перечня, слой журналов и специальностей, 
	и слой связей между сущностями. Описаны основные сущности предметной области: Journal (Журнал), Specialty (Специальность), 
	JournalVersion (Версия перечня), JournalSpeciality (Включение журнала в перечень), ArticleValidation (Валидация статьи). 
	Представлена ER-диаграмма концептуальной модели с описанием атрибутов сущностей, первичных и внешних ключей, 
	связей между сущностями. Описаны семантические ограничения целостности данных, обеспечивающие корректность хранения 
	информации о версионности перечней и временных интервалах включения журналов.
\end{annotation}

\section{Семантические связи между сущностями предметной области}

\begin{annotation}
	Раздел описывает семантические связи между сущностями предметной области и их представление в модели данных. 
	Представлены UML-диаграммы классов, отражающие отношения между сущностями: ассоциации, агрегации и композиции. 
	Описаны семантические связи между журналами и специальностями через сущность JournalSpeciality с указанием временных 
	интервалов включения/исключения. Представлены связи между версиями перечня и журналами, обеспечивающие поддержку 
	версионности данных. Описаны связи между статьями и результатами их валидации через сущность ArticleValidation. 
	Представлены алгоритмы работы с семантическими связями для ответа на запросы о статусе журнала на конкретную дату 
	и валидации публикаций с учётом временных аспектов данных.
\end{annotation}