\chapter{Проектирование системы автоматической валидации научных публикаций}

В настоящей главе представлены детальные результаты проектирования системы автоматической валидации научных публикаций. 
Сформулированы функциональные требования, включающие возможности загрузки и парсинга перечней ВАК, валидации публикаций по 
ISSN и дате, поиска журналов по специальностям, просмотра истории изменений перечня. Разработаны диаграммы вариантов использования 
(use case), описывающие сценарии работы различных категорий пользователей. 
Представлена трёхуровневая архитектура системы (presentation layer, business logic layer, data layer) с детализацией компонентов 
каждого уровня и протоколов их взаимодействия. 
Спроектирована физическая модель базы данных, включающая схему таблиц с индексами, внешними ключами и триггерами для 
поддержки версионности и обеспечения целостности данных. Обоснован выбор технологического стека: Python 3.11+ с фреймворком FastAPI 
для реализации REST API, React для frontend-части сервиса, а также выбор СУБД PostgreSQL. Детально описаны проектные решения 
по организации модульной архитектуры backend с применением паттернов Dependency 
Injection, Repository и Service Layer, обеспечивающих тестируемость и поддерживаемость кода.