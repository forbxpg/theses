\chapter*{Введение}
\label{sec:afterwords}
\addcontentsline{toc}{chapter}{Введение}

\subsubsection*{Актуальность работы}
В современной системе высшего образования и научных исследований Российской Федерации
проверка соответствия научных публикаций официальному перечню рецензируемых 
изданий Высшей аттестационной комиссии (ВАК) \cite{VAK} является критически важной
задачей. Эта процедура необходима как при аттестации научных кадров, так и при оценке
исследовательской деятельности вузов и научных организаций.
Однако текущий процесс валидации полностью основан на ручной проверке через веб-портал, 
где необходимо последовательно искать каждый журнал в перечне. Это создаёт существенные трудности, а именно:

\begin{itemize}
	\item \textbf{Отсутствие API:} перечень доступен исключительно в виде
	PDF-документов, что исключает возможность автоматизированной интеграции;
    \item \textbf{Динамичность данных:} перечень обновляется каждые 1–3 месяца, 
	и необходимо отслеживать исторические версии для проверки старых публикаций;
\end{itemize}

\subsubsection*{Нерешенные проблемы}

Анализ предметной области выявил следующие критические пробелы:

\begin{itemize}
	\item Отсутствие стандартизированной структуры данных для хранения информации о перечнях на конкретные даты;
	\item Невозможность надёжного автоматического парсинга PDF из-за вариативности структуры и форматирования документов;
	\item Отсутствие механизма версионирования с отслеживанием изменений в составе перечня;
	\item Недостаток информации о точных датах включения и исключения журналов;
	\item Наличие аномалий в данных, затрудняющих корректную валидацию;
	\item Необходимость интеграции с внешними сервисами (такими как CrossRef API или Arxiv API) для получения метаданных публикаций.
\end{itemize}

\subsubsection*{Научная новизна работы}

Научная новизна данного исследования заключается в преодолении фундаментальных проблем неструктурированных, 
динамически изменяющихся и зашумленных данных в контексте научной метрологии и информационного обеспечения науки. 
В отличие от существующих решений, предлагаемая работа базируется на следующих научных положениях и подходах:

\begin{compactenum}
	\item \textbf{Разработка формальной онтологии и концептуальной модели для представления динамических нормативных перечней.} 
		Впервые предлагается многоаспектная модель данных, которая не просто фиксирует статичный список объектов, 
		а описывает жизненный цикл научного журнала в контексте перечня ВАК. 
		Модель учитывает временную размерность (включение, исключение), 
		статусную атрибутику и связи между различными версиями перечня, что позволяет однозначно определять 
		статус журнала на любую историческую дату и решает проблему «версионного хаоса».
	\item \textbf{Разработка алгоритмов временного логического вывода для валидации научных публикаций.} 
		В работе предлагаются формальные алгоритмы, которые на основе построенной онтологии и временной метки публикации 
		выполняют проверку её соответствия не просто актуальному списку, а той версии перечня, которая действовала на момент 
		публикации. Этот подход решает ключевую научно-практическую проблему ретроспективной валидации, которая ранее не могла 
		быть решена в автоматизированном режиме из-за отсутствия связанных исторических данных
\end{compactenum}

\subsubsection*{Содержание по главам}
Работа состоит из следующих ключевых разделов:

\begin{compactenum}
	\item \textbf{Глава 1:} анализ существующих методов парсинга, 
	сравнение подходов к концептуализации предметной области, выбор концептуальной модели, 
	технологического стека и их обоснование, постановка задач для реализации.
\end{compactenum}

\subsubsection*{Практическая значимость работы}
Разработанная система может быть использована вузами, научными учреждениями, издательствами 
и организациями РАН для автоматизации и оптимизации процесса проверки 
соответствия публикаций перечню ВАК, что существенно снижает временные затраты
и риск ошибок.


%%% Local Variables:
%%% TeX-engine: xetex
%%% eval: (setq-local TeX-master (concat "../" (seq-find (-cut string-match ".*-3-pz\.tex$" <>) (directory-files ".."))))
%%% End:
