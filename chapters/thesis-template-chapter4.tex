\chapter{Программная реализация и экспериментальная проверка системы автоматической валидации научных публикаций}

В настоящей главе представлены результаты практической реализации и экспериментальной проверки разработанной системы. 
Описан состав и структура реализованного программного обеспечения: backend-сервис на FastAPI, 
включающий модули парсинга PDF-документов, работы с базой данных через SQLAlchemy ORM, интеграции с CrossRef API и REST API 
эндпоинты; frontend-приложение на React с компонентами валидации публикаций, поиска журналов, просмотра истории перечней и 
административной панели; база данных PostgreSQL с реализованной схемой и системой миграций. Представлены основные сценарии работы 
пользователей с описанием интерфейсов: для исследователей — форма валидации публикации с получением мгновенного результата 
проверки; для администраторов — загрузка и обработка PDF-перечней, мониторинг системы; для внешних систем — интеграция через REST API с
 примерами запросов и ответов. Разработаны тестовые примеры, включающие модульные тесты для алгоритмов парсинга и валидации, 
 интеграционные тесты для проверки взаимодействия компонентов.

\section{Описание реализации}

\begin{annotation}
	Раздел содержит детальное описание практической реализации системы автоматической валидации научных публикаций. 
	Представлена архитектура реализованного программного обеспечения с описанием компонентов backend-части на FastAPI: 
	модули парсинга PDF-документов перечней ВАК, работа с базой данных через SQLAlchemy ORM с применением паттерна Unit of Work, 
	интеграция с CrossRef API для получения метаданных статей, REST API эндпоинты для валидации публикаций и управления системой. 
	Описана реализация frontend-приложения на React: компоненты для валидации публикаций, отображения результатов валидации, 
	пагинированного списка валидаций пользователя, аутентификации и авторизации. Представлена структура базы данных PostgreSQL 
	с описанием таблиц, индексов, внешних ключей и триггеров. Описаны основные сценарии работы пользователей с примерами 
	интерфейсов и взаимодействия с системой. Приведены примеры кода ключевых компонентов системы с пояснениями реализации.
\end{annotation}

\section{Тестирование}

\begin{annotation}
	Раздел посвящён описанию процесса тестирования системы автоматической валидации научных публикаций. 
	Представлена стратегия тестирования, включающая модульные тесты для проверки отдельных компонентов системы 
	(алгоритмы парсинга PDF-документов, логика валидации публикаций, работа с базой данных), интеграционные тесты 
	для проверки взаимодействия между компонентами (интеграция с CrossRef API, работа с базой данных через Unit of Work, 
	взаимодействие между backend и frontend), тесты производительности для оценки времени отклика API и пропускной способности системы. 
	Описаны используемые инструменты тестирования (pytest для Python, Jest для React), подходы к мокированию внешних зависимостей 
	и созданию тестовых данных. Приведены результаты тестирования с показателями покрытия кода тестами, описанием выявленных 
	ошибок и их исправления. Представлены примеры тестовых сценариев для критических функций системы.
\end{annotation}

\section{Развитие в будущем}

\begin{annotation}
	Раздел содержит описание направлений дальнейшего развития системы автоматической валидации научных публикаций. 
	Рассмотрены потенциальные улучшения функциональности: расширение возможностей парсинга для работы с различными форматами 
	перечней ВАК, добавление поддержки валидации публикаций по дополнительным критериям, интеграция с другими внешними API 
	для получения метаданных публикаций, реализация автоматического обновления перечней ВАК через планировщик задач. 
	Представлены предложения по улучшению производительности: оптимизация запросов к базе данных, кэширование часто используемых данных, 
	масштабирование системы для обработки больших объёмов запросов. Описаны возможности расширения архитектуры: реализация 
	микросервисной архитектуры, добавление очередей сообщений для асинхронной обработки задач, интеграция с системами мониторинга и логирования. 
	Обсуждены перспективы интеграции системы с информационными системами вузов и научных учреждений.
\end{annotation}