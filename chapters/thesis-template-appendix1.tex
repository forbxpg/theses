\chapter{Основные правила форматирования}\label{app-format}
%\addcontentsline{toc}{chapter}{}

Текст пояснительной записки должен готовиться для печати на листах формата А4, использоваться должен шрифт с засечками (Roman; обычно --- Times Roman или Times New Roman), 12 или 14 кегль. Размеры полей:

\begin{itemize}
	\item верхнее: 20 мм.
	\item нижнее: 20 мм.
	\item левое: 10 мм.
	\item правое: 25 мм.
\end{itemize}

Нумероваться должны все страницы, начиная с первой (титульной), однако сами номера следует проставлять на страницах, начиная со страницы реферата. Номер следует проставлять внизу страницу (в центре).

Заголовки оформляются тем же шрифтом, что и основной текст (т.е., соответственно, Times Roman или Times New Roman). Для заголовков первого уровня размер шрифта может быть больше размера шрифта основного текста (обычно 14-16).

Все разделы текста: реферат, оглавление, введение, три главы основного
содержания, список литературы, заключение, приложения --- должны снабжаться
содержательным заголовком и начинаться с новой страницы; сами заголовки следует
при этом центрировать (заголовки параграфов и пунктов выравниваются по ширине).
Следует обратить внимание, что заголовки всех разделов, кроме трех основных
глав, регламентированы; заголовки трех основных глав должны быть содержательными
и отражать суть соответствующей главы. Названия типа <<Аналитическая часть>> и <<Теоретическая глава>> --- \textit{недопустимы}.

Текст пояснительной записки может содержать рисунки и таблицы. Все рисунки и
таблицы должны снабжаться номерами и подписями:

\begin{itemize}

	\item нумерация рисунков и таблиц должна быть сквозная (но раздельная, т.к. для рисунков своя, для таблиц --- своя);

	\item в случае большого количества иллюстраций/таблиц, допускается <<вложенная>> нумерация (т.е. таблицу/рисунок можно снабжать составным номером в формате 
	
	$$\langle\mbox{номер главы}\rangle.\langle\mbox{номер внутри главы}\rangle;$$
	
	\item подрисуночная подпись должна располагаться снизу по центру;
	
	\item название таблицы следует помещать над таблицей слева, без абзацного
	отступа в одну строку с ее номером через тире (ГОСТ 7.32-2001, п.6.6.1).

\end{itemize}

Здесь перечислены не все, а лишь основные требования к оформлению. Прочие
требования --- см. соответствующие ГОСТы.

Для того чтобы избежать больших отступов в списках, которые по умолчанию добавляют окружения \texttt{itemize} и \texttt{enumerate}, следует использовать 
\texttt{compactitem} (для маркированных списков) и \texttt{compactenum} (для нумерованных списков) из пакета \texttt{paralist}. 
Например:

\begin{compactitem}
	\item это;
	\item не нумерованный;
	\item список;
	\item без лишних промежутков.
\end{compactitem}

И для нумерованных списков:

\begin{compactenum}[1)]
	\item нумерованные списки;
	\item пакета \texttt{paralist};
	\item еще и удобно настраивать;
	\item (например, менять формат номера).
\end{compactenum}

\noindent или

\begin{compactenum}[a)]
	\item это другой;
	\item нумерованный;
	\item список;
	\item без лишних промежутков;
	\item и с буквенной нумерацией.
\end{compactenum}

А если хочется нумерацию сделать ангоязычной, то нужно использовать окружение \texttt{other\-language} (таким образом: \verb|\begin{otherlanguage}[numerals=latin]{russian}|)

\setkeys{russian}{numerals=latin}
%\selectlanguage{russian}
%\begin{otherlanguage}[numerals=latin]{russian}
\begin{russian}
\begin{compactenum}[a)]
	\item это другой;
	\item нумерованный;
	\item список;
	\item без лишних промежутков;
	\item и с буквенной нумерацией.
\end{compactenum}
\end{russian}
%\end{otherlanguage}

\textbf{Замечание.} По неизвестным причинам, переключения не происходит, хотя должно.

%%% Local Variables:
%%% TeX-engine: xetex
%%% eval: (setq-local TeX-master (concat "../" (seq-find (-cut string-match ".*-3-pz\.tex$" <>) (directory-files ".."))))
%%% End:
